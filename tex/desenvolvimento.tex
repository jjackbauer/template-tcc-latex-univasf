%--------------------------------------------------------------------------------------
% Este arquivo contém a sua funtamentação teórica
%--------------------------------------------------------------------------------------
\chapter{Referencial teórico}
\section{Ondas Cerebrais}
%https://en.wikipedia.org/wiki/Neural_oscillation
\section{Neuroimagem}
\section{Ruido}
\section{Amostragem}
\subsection{Ruido em decorrência da amostragem}
\section{Quantização}
\subsection{Ruido em decorrência da quantização}
\subsection{Teorema da Amostragem de Nyquist–Shannon}
\subsection{Downsampling ou Decimação}
\section{Razão Sinal-Ruido}
\section{Instrumentação Eletrônica}
\section{Magnetômetro}
\section{Gradiômetro Planar}
\section{Magneto-Encefalografia}
\section{Interface Cérebro-Máquina}
\section{Validação de modelos de aprendizagem}
\subsection{Cross Validation}
\subsection{Leave-one-out}
\subsection{Nested Cross Validation}
\subsection{Repeated Random Splits}
\section{Bias-Variance Tradeoff}



















\begin{comment}
\section{Seção de exemplo 1 - Como fazer citações}

Existem vários tipos de citações...


\section{Seção de exemplo 2 - Como inserir figuras}

Neste trabalho iremos exemplificar duas formas de se inserir figuras no Latex. O primeiro método insere, no documento, uma figura simples por meio do comando:

\textbackslash imagem\{ Escala \}\{ Arquivo sem extensão \}\{ Descrição \}\{ Fonte \}

\textbf{Obs.:} A fonte pode ser uma citação do tipo  \textbackslash citeonline\{\}.

A figura \ref{img:placeholder} é um exemplo deste método.

%--------------------------------------------------------------------------------------
% Esse é um exemplo de figura simples
%--------------------------------------------------------------------------------------
\imagem{0.15}{placeholder}{Uma figura simples}{O autor}

A figura \ref{img:figura1} é um exemplo do outro tipo de figura abordada aqui, chamada de figura composta. Esta figura é composta de outras subfiguras.
%--------------------------------------------------------------------------------------
% Esse é um exemplo de figura composta de outras subfiguras
%--------------------------------------------------------------------------------------
\begin{figure}[!htb]
\centering
    \caption{\label{img:figura1} Exemplo de figura composta}
    \subcaptionbox{\label{img:subfigura1} Subfigura 1}{\includegraphics[scale=.1]{img/placeholder}}\qquad
    \subcaptionbox{\label{img:subfigura2} Subfigura 2}{\includegraphics[scale=.1]{img/placeholder}}
    \vspace{1.5em}
    \legend{\textbf{Fonte:} \citeonline{SUA-REFERENCIA}}
\label{fig:dag}
\end{figure}

%\begin{figure}[!htb]
%\centering
%    \caption{\label{img:telas} Telas da aplicação cliente}
%    \subcaptionbox{\label{img:inicial} Abertura}{\includegraphics[scale=.12]{img/APP/inicial}}\qquad
%    \subcaptionbox{\label{img:login} \textit{Login}}{\includegraphics[scale=.12]{img/APP/login}}\qquad
%    \subcaptionbox{\label{img:cadastro} Cadastro}{\includegraphics[scale=.12]{img/APP/cadastro}}\qquad
%    \subcaptionbox{\label{img:hist-rel}Sobre}{\includegraphics[scale=.12]{img/APP/sobre}}\\
%    \vspace{1.5em}
%    \subcaptionbox{\label{img:dados_atuais}Dados atuais}{\includegraphics[scale=.15]{img/APP/atual}}\qquad
%    \subcaptionbox{\label{img:hist-time}Seleção de período}{\includegraphics[scale=.15]{img/APP/periodo}}\qquad
%    \subcaptionbox{\label{img:hist-rel}Exibir histórico}{\includegraphics[scale=.15]{img/APP/historico}}\\
%    \vspace{2.5em}
%    \legend{\textbf{Fonte:} O Autor}
%\label{fig:dag}
%\end{figure}

Para referenciar uma figura deve ser usada comando \textbackslash ref\{img:<label ou nome do arquivo>\}, como exemplo, estamos referenciando a figura \ref{img:placeholder}. Isso vale tanto para figuras simples quanto para as compostas, como por exemplo as figuras \ref{img:subfigura1} e \ref{img:subfigura2}. Ao inserir uma figura, ela é automaticamente identificada e incluída no elemento pré-textual da lista de figuras.




\section{Seção de exemplo 3 - Sobre tabelas}

As tabelas em Latex são deveras capciosas, por isso não serão abordadas em sua completude neste documento.

Há um site que possui uma ferramenta interessante para ser utilizada na construção tabelas em Latex.

\centerline{\href{https://www.tablesgenerator.com/}{ O Tables Generator } <-- Isto é um \textit{link} :D}

Contudo, busquem entendimento sobre o assunto, pois tabelas são elementos textuais importantes e enriquecem muito o texto, quando bem construídas.

A tabela \ref{tab:crossplatform} é um exemplo de como uma tabela pode ser construída, assim como a tabela do anexo \ref{anex:anexo1}.

\begin{table}[!htb]
	\centering
	\caption{\label{tab:crossplatform} Tipos de aplicações e abordagens preferenciais.}
	\begin{adjustbox}{max width=\textwidth}
		\begin{tabular}{@{} p{5cm} ccc @{}}
		\toprule
		\textbf{Código da Aplicação} & \textbf{Web} & \textbf{Híbrida} & \textbf{Interpretada / Compilação Cruzada} \\ \hline

		\textbf{Aplicações baseadas em dados providos por um servidor} &
			3 & 2 & 1
		\\ \hline

		\textbf{Aplicações independentes} & 1 & 2 & 3\\ \hline

		\textbf{Aplicações baseadas em sensores e processamento de dados no dispositivo} & 1 & 2 & 3\\ \hline

		\textbf{Aplicações baseadas em sensores e processamento de dados no servidor} & 1 & 3 & 2\\ \hline

		\textbf{Aplicações Cliente-Servidor} & 1 & 3 & 2 \\ \bottomrule
	\end{tabular}
	\end{adjustbox}
	\legend{\textbf{Fonte:} \citeonline{raj2012study} (Traduzido)}
\end{table}

Também é possível criar quadros, que são ligeiramente diferente de tabelas. Acompanhe o exemplo no Quadro \ref{qua:confusionmatrix}

\begin{quadro}
	\centering
	\caption{\label{qua:confusionmatrix}Exemplo de matriz de confusão}
	\begin{tabular}{ll|c|c|}
		\cline{3-4}
		\multicolumn{1}{c}{\textbf{}} & \multicolumn{1}{c|}{\textbf{}} & \multicolumn{2}{l|}{\textbf{Classe prevista}} \\ \cline{3-4}
		 & \multicolumn{1}{c|}{\textbf{}} & Classe = 1 & Classe = 0 \\ \hline
		\multicolumn{1}{|l|}{\multirow{2}{*}{\textbf{Classe real}}} & Classe = 1 & $f_{11}$ & $f_{10}$ \\ \cline{2-4}
		\multicolumn{1}{|l|}{} & Classe = 0 & $f_{01}$ & $f_{00}$ \\ \hline
	\end{tabular}
	\Ididthis
\end{quadro}

\section{Subseção de exemplo 4 - Seções}
\end{comment}

