\documentclass[
	12pt,				% tamanho da fonte
	openany,			% capítulos começam em pág ímpar (insere página vazia caso preciso)
	oneside, 			% oneside - twoside
	a4paper,			% tamanho do papel.
	chapter=TITLE,		% títulos de capítulos convertidos em letras maiúsculas
	section=TITLE,		% títulos de seções convertidos em letras maiúsculas
	sumario=tradicional,	
	%subsection=TITLE,	% títulos de subseções convertidos em letras maiúsculas
	%subsubsection=TITLE,% títulos de subsubseções convertidos em letras maiúsculas
	english,			% idioma adicional para hifenização
	brazil,				% o último idioma é o principal do documento
	]{abntex2}

% ---------------------------------------------------------------------------
% Inclui os comandos do projeto
% ---------------------------------------------------------------------------
\input{tex/commands}

% ---------------------------------------------------------------------------
% IDENTIFICAÇÃO
% ---------------------------------------------------------------------------
\titulo{\MakeUppercase{Avaliação de métricas de validação de decodificadores de sinais de magneto-encefalograma entre indivíduos}}
\autor{RICARDO VALÉRIO TEIXEIRA DE MEDEIROS SILVA}
\local{JUAZEIRO - BA}
\orientador{Prof. Dr. Rosalvo Ferreira de Oliveira Neto}
%\coorientador{Prof. Dr Rodrigo Pereira Ramos}
\instituicao{
UNIVERSIDADE FEDERAL DO VALE DO SÃO FRANCISCO
	\par
CURSO DE GRADUAÇÃO EM ENGENHARIA DE COMPUTAÇÃO}
\tipotrabalho{Trabalho de Conclusão de Curso}
\preambulo{Trabalho apresentado à Universidade Federal do Vale do São Francisco - Univasf, Campus Juazeiro, como requisito da obtenção do título de Bacharel em Engenharia de Computação.}


% -----------------------------------------------------------------------------
% CONFIGURACAO DO SUMARIO - by @Gabrielr2508
% Precisa estar aqui, por isso não foi para o commands.tex, não descobrimos o motivo, %caso saiba, por favor, faça um pull request! :D
% -----------------------------------------------------------------------------
% Secao primaria (Chapter) Caixa alta, Negrito, tamanho 12
\makeatletter
\renewcommand*{\l@chapter}[2]{%
  \l@chapapp{\uppercase{#1}}{#2}{\cftchaptername}}
\makeatother
% Secao secundaria (Section) Caixa baixa, Negrito, tamanho 12
\renewcommand{\cftsectionfont}{\uppercase} %ponha \rmfamily se quiser serifadas...

% Secao terciaria (Subsection) Caixa baixa, negrito, tamanho 12
\renewcommand{\cftsubsectionfont}{\bfseries}

% Secao quaternaria (Subsubsection) Caixa baixa, tamanho 12
\renewcommand{\cftsubsubsectionfont}{\normalfont}

% Seção quinaria (subsubsubsection) Caixa baixa, sem negrito, tamanho 12
\renewcommand{\cftparagraphfont}{\normalfont\itshape}

% -----------------------------------------------------------------------------
% Início do TCC 
% -----------------------------------------------------------------------------
\begin{document}

	\frenchspacing % Retira espaço extra obsoleto entre as frases.
	
	\pretextual
		\input{tex/pretextual}

	\textual
		\pagestyle{simple}
		%--------------------------------------------------------------------------------------
% Este arquivo contém a sua introdução, objetivos e organização do trabalho
%--------------------------------------------------------------------------------------
\chapter{Introdução}


%Colocar citações longas entre \textbackslash begin\{citacao\} e \textbackslash end\{citacao\}, exemplo: 

%\begin{citacao}
%``\lipsum[1]''
%
%\cite{REFERENCIA}
%\end{citacao}

\section{Motivação}
A decodificação da atividade cerebral tem diversas aplicações, dentre elas, estudos neurológicos e Interfaces Cérebro-Máquina (ICMs).
As interfaces Cérebro-Máquina traduzem a atividade cerebral do usuário em tempo real para comandos destinados a um dispositivo externo. as abordagens não invasivas tem diversas aplicações como videogames, sistemas de comunicação ou reabilitação neurológica de pessoas com deficiência,\cite{halme2018across}.Contudo, segundo \cite{corsi2019integrating} ainda há muita incerteza na forma de se identificar uma estratégia para codificar a intenção do usuário e otimização das características de controle. Estes desafios são ainda mais evidentes quando, segundo \cite{vidaurre2010towards}, a taxa de usuários que não conseguem utilizar ICMs de forma adequada é de até 30\%, afetando principalmente as abordagens baseadas em Imagem Motora.


Ainda assim, as variações estruturais e funcionais do cérebro de indivíduos distintos torna árdua a tarefa de decodificar atividade cerebral entre diferentes sujeitos. Isso faz com que a abordagem de treinar um classificador com um grupo de indivíduos e depois testar com um grupo disjunto deste seja muitas vezes ineficiente como discutido em \cite{olivetti2014meg} e  \cite{Barachant}. Outra questão particular ao problema de criar e validar modelos para decodificação da atividade cerebral é a dificuldade de se obter leituras com boa razão sinal-ruído (SNR), o que torna os bancos de dados disponíveis escassos em relação à grande dimensionalidade dos sinais de interesse, \cite{varoquaux2017assessing}.

O estado da arte da decodificação de sinais de magneto-encefalograma entre indivíduos ainda tem uma precisão muito baixa se comparada com outras aplicações que se utilizam se inteligência artificial. Essa baixa precisão faz com que as ICMs atuais ainda sejam em sua maioria protótipos testados em laboratórios, \cite{clerc2016brain2}. Logo, é de grande relevância o estudo de técnicas de validação dos classificadores utilizados, pois estas técnicas permitirão ter mais discernimento sobre o poder preditivo dos classificadores criados para decodificar sinais de MEG ou outros modos de imagem cerebral \cite{varoquaux2017assessing}.


\section{Definição do problema}

Dado um sinal magneto-encefalografia (MEG)$\overline{X}$, que é a amostra da atividade eletromagnética do cérebro de um indivíduo, capturado após um estímulo dentre duas classes diferentes. Este sinal é composto de vários canais, que cobrem à atividade cerebral da pessoa em uma malha, cujos nós medem a intensidade e variação do campo eletromagnético induzido pela atividade cerebral do indivíduo. A grandeza dos valores medidos é muito pequena, portanto os aparelhos devem ter uma alta sensibilidade (o que ocasiona uma baixa razão sinal-ruído na maioria das aquisições). Além disto, por causa dos vários canais e alta frequência de amostragem, os sinais de MEG tem uma alta dimensionalidade, o que torna mais difícil a obtenção de conhecimento através dos sinais de MEG.

A abordagem mais comum para a decodificação de sinais de magneto-encefalografia consiste dos seguintes passos segundo \cite{nam2018brain}:
\begin{itemize}
    \item preprocessamento para diminuição de ruídos e decimação do sinal
    \item extração de características mais relevantes do sinal
    \item Criação de modelos de classificação
    \item Validação de modelos de classificação
    \item Classificação das instâncias não rotuladas
\end{itemize}

Todavia, validar modelos criados para decodificação de neuroimagem é uma tarefa árdua, pois há uma escassez de dados em relação à dimensionalidade inerente destes sistemas \cite{carroll2009prediction}. Este problema é ainda maior quando se trata da criação de um modelo de classificação que deve classificar a neuroimagem de indivíduos distintos. Geralmente, para obter a uma precisão maior de classificação é utilizada à técnica do Leave-one-out, \cite{Barachant} e \cite{olivetti2014meg}. Contudo, segundo \cite{varoquaux2017assessing} esta técnica leva a resultados instáveis e precisões maiores podem ser obtidas utilizando \textit{Repeated random splits}, divisões aleatórias repetidas, para validar aos classificadores treinados. 

\section{Objetivos gerais}
Avaliar o efeito da escolha de algoritmos de validação de modelos e classificação na decodificação de sinais de MEG em bases de dados abertas e previamente utilizadas na literatura.

\section{Objetivos específicos}
\begin{itemize}
\item Avaliar o uso do Holdout como algoritmo de validação para decodificação de sinais de MEG entre indivíduos.
\item Avaliar o uso do Leave-one-out como algoritmo de validação para decodificação de sinais de MEG entre indivíduos.
\item Avaliar o uso do Repeated random splits como algoritmo de validação para decodificação de sinais de MEG entre indivíduos.
\item Avaliar o uso do Regressão Logistica como algoritmo de classificação para decodificação de sinais de MEG entre indivíduos.
\item Avaliar o uso do Random Forrest como algoritmo de classificação para decodificação de sinais de MEG entre indivíduos.
\item Avaliar o uso do Support Vector Machines lineares e não-lineares como algoritmo de classificação para decodificação de sinais de MEG entre indivíduos.

\end{itemize}

\section{Organização do trabalho}

%\lipsum[10-12]

		%--------------------------------------------------------------------------------------
% Este arquivo contém a sua funtamentação teórica
%--------------------------------------------------------------------------------------
\chapter{Referencial teórico}
\section{Ondas Cerebrais}
%https://en.wikipedia.org/wiki/Neural_oscillation
\section{Neuroimagem}
\section{Ruido}
\section{Amostragem}
\subsection{Ruido em decorrência da amostragem}
\section{Quantização}
\subsection{Ruido em decorrência da quantização}
\subsection{Teorema da Amostragem de Nyquist–Shannon}
\subsection{Downsampling ou Decimação}
\section{Razão Sinal-Ruido}
\section{Instrumentação Eletrônica}
\section{Magnetômetro}
\section{Gradiômetro Planar}
\section{Magneto-Encefalografia}
\section{Interface Cérebro-Máquina}
\section{Validação de modelos de aprendizagem}
\subsection{Cross Validation}
\subsection{Leave-one-out}
\subsection{Nested Cross Validation}
\subsection{Repeated Random Splits}
\section{Bias-Variance Tradeoff}



















\begin{comment}
\section{Seção de exemplo 1 - Como fazer citações}

Existem vários tipos de citações...


\section{Seção de exemplo 2 - Como inserir figuras}

Neste trabalho iremos exemplificar duas formas de se inserir figuras no Latex. O primeiro método insere, no documento, uma figura simples por meio do comando:

\textbackslash imagem\{ Escala \}\{ Arquivo sem extensão \}\{ Descrição \}\{ Fonte \}

\textbf{Obs.:} A fonte pode ser uma citação do tipo  \textbackslash citeonline\{\}.

A figura \ref{img:placeholder} é um exemplo deste método.

%--------------------------------------------------------------------------------------
% Esse é um exemplo de figura simples
%--------------------------------------------------------------------------------------
\imagem{0.15}{placeholder}{Uma figura simples}{O autor}

A figura \ref{img:figura1} é um exemplo do outro tipo de figura abordada aqui, chamada de figura composta. Esta figura é composta de outras subfiguras.
%--------------------------------------------------------------------------------------
% Esse é um exemplo de figura composta de outras subfiguras
%--------------------------------------------------------------------------------------
\begin{figure}[!htb]
\centering
    \caption{\label{img:figura1} Exemplo de figura composta}
    \subcaptionbox{\label{img:subfigura1} Subfigura 1}{\includegraphics[scale=.1]{img/placeholder}}\qquad
    \subcaptionbox{\label{img:subfigura2} Subfigura 2}{\includegraphics[scale=.1]{img/placeholder}}
    \vspace{1.5em}
    \legend{\textbf{Fonte:} \citeonline{SUA-REFERENCIA}}
\label{fig:dag}
\end{figure}

%\begin{figure}[!htb]
%\centering
%    \caption{\label{img:telas} Telas da aplicação cliente}
%    \subcaptionbox{\label{img:inicial} Abertura}{\includegraphics[scale=.12]{img/APP/inicial}}\qquad
%    \subcaptionbox{\label{img:login} \textit{Login}}{\includegraphics[scale=.12]{img/APP/login}}\qquad
%    \subcaptionbox{\label{img:cadastro} Cadastro}{\includegraphics[scale=.12]{img/APP/cadastro}}\qquad
%    \subcaptionbox{\label{img:hist-rel}Sobre}{\includegraphics[scale=.12]{img/APP/sobre}}\\
%    \vspace{1.5em}
%    \subcaptionbox{\label{img:dados_atuais}Dados atuais}{\includegraphics[scale=.15]{img/APP/atual}}\qquad
%    \subcaptionbox{\label{img:hist-time}Seleção de período}{\includegraphics[scale=.15]{img/APP/periodo}}\qquad
%    \subcaptionbox{\label{img:hist-rel}Exibir histórico}{\includegraphics[scale=.15]{img/APP/historico}}\\
%    \vspace{2.5em}
%    \legend{\textbf{Fonte:} O Autor}
%\label{fig:dag}
%\end{figure}

Para referenciar uma figura deve ser usada comando \textbackslash ref\{img:<label ou nome do arquivo>\}, como exemplo, estamos referenciando a figura \ref{img:placeholder}. Isso vale tanto para figuras simples quanto para as compostas, como por exemplo as figuras \ref{img:subfigura1} e \ref{img:subfigura2}. Ao inserir uma figura, ela é automaticamente identificada e incluída no elemento pré-textual da lista de figuras.




\section{Seção de exemplo 3 - Sobre tabelas}

As tabelas em Latex são deveras capciosas, por isso não serão abordadas em sua completude neste documento.

Há um site que possui uma ferramenta interessante para ser utilizada na construção tabelas em Latex.

\centerline{\href{https://www.tablesgenerator.com/}{ O Tables Generator } <-- Isto é um \textit{link} :D}

Contudo, busquem entendimento sobre o assunto, pois tabelas são elementos textuais importantes e enriquecem muito o texto, quando bem construídas.

A tabela \ref{tab:crossplatform} é um exemplo de como uma tabela pode ser construída, assim como a tabela do anexo \ref{anex:anexo1}.

\begin{table}[!htb]
	\centering
	\caption{\label{tab:crossplatform} Tipos de aplicações e abordagens preferenciais.}
	\begin{adjustbox}{max width=\textwidth}
		\begin{tabular}{@{} p{5cm} ccc @{}}
		\toprule
		\textbf{Código da Aplicação} & \textbf{Web} & \textbf{Híbrida} & \textbf{Interpretada / Compilação Cruzada} \\ \hline

		\textbf{Aplicações baseadas em dados providos por um servidor} &
			3 & 2 & 1
		\\ \hline

		\textbf{Aplicações independentes} & 1 & 2 & 3\\ \hline

		\textbf{Aplicações baseadas em sensores e processamento de dados no dispositivo} & 1 & 2 & 3\\ \hline

		\textbf{Aplicações baseadas em sensores e processamento de dados no servidor} & 1 & 3 & 2\\ \hline

		\textbf{Aplicações Cliente-Servidor} & 1 & 3 & 2 \\ \bottomrule
	\end{tabular}
	\end{adjustbox}
	\legend{\textbf{Fonte:} \citeonline{raj2012study} (Traduzido)}
\end{table}

Também é possível criar quadros, que são ligeiramente diferente de tabelas. Acompanhe o exemplo no Quadro \ref{qua:confusionmatrix}

\begin{quadro}
	\centering
	\caption{\label{qua:confusionmatrix}Exemplo de matriz de confusão}
	\begin{tabular}{ll|c|c|}
		\cline{3-4}
		\multicolumn{1}{c}{\textbf{}} & \multicolumn{1}{c|}{\textbf{}} & \multicolumn{2}{l|}{\textbf{Classe prevista}} \\ \cline{3-4}
		 & \multicolumn{1}{c|}{\textbf{}} & Classe = 1 & Classe = 0 \\ \hline
		\multicolumn{1}{|l|}{\multirow{2}{*}{\textbf{Classe real}}} & Classe = 1 & $f_{11}$ & $f_{10}$ \\ \cline{2-4}
		\multicolumn{1}{|l|}{} & Classe = 0 & $f_{01}$ & $f_{00}$ \\ \hline
	\end{tabular}
	\Ididthis
\end{quadro}

\section{Subseção de exemplo 4 - Seções}
\end{comment}


		%--------------------------------------------------------------------------------------
% Este arquivo contém a sua metodologia
%--------------------------------------------------------------------------------------
\chapter{Materiais e Métodos} \label{ch:MM} %Uma label é como você referencia uma seção no texto com a tag \ref{}


\section{Base de dados}
A base de dados utilizada neste trabalho foi a disponibilizada pela competição DecMeg2014, \cite{KagleComp}, cujo Objetivo era prever estímulos visuais através de gravações de MEG da atividade cerebral. 

Os dados utilizados para este trabalho consistem em 9414 coletas de MEG para treino, realizadas em 16 indivíduos diferentes, com a indicação de qual estímulo o indivíduo em questão foi exposto, face ou face embaralhada. E também 4058 coletas para teste, realizadas em 7 indivíduos, sem as respectivas classes de estimulo descriminadas. Cada sujeito tem aproximadamente entre 580 a 590 coletas disponíveis para ambos os grupos. 

A coleta é constituída de 1.5 segundos de gravação do MEG, começando 0.5 segundos antes do estímulo começar. Para os dados de treino, inclui-se também a classe do respectivo estimulo, face (1) ou face embaralhada (0). Originalmente os dados foram coletados numa frequência de 1 Khz, contudo os dados disponibilizados foram sub-amostrados para frequência  de 250 hz e um filtro passa alta foi aplicado em 1hz para eliminar componentes contínuas (CC) que não compõe o sinal de interesse.

Os procedimentos descritos anteriormente foram realizados com a biblioteca nme-python e os dados foram arranjados em uma matriz tridimensional que indexa Coleta x Canal x Tempo, com tamanho de  580 - 590 x 306 x 375. Para o conjunto de treino, esta matriz está associada a um vetor cuja dimensão é igual à primeira dimensão da matriz, no qual estão às \textit{labels} associadas.

Os sensores utilizados foram espalhados como demonstrado na (Imagem), agrupados em grupos de três sensores por local, cobrindo 102 locais. Em cada local há dois gradiômetros ortogonais e um magnetômetro que fazem a gravação do campo magnético induzido pelas ondas cerebrais. Os gradiômetros medem a variação espacial do campo magnético, dando origem ao plano XY e o magnetômetro mede a componente radial Z do mesmo.   


%\subsection{Subseção de exemplo 1 - Referenciando seções} \label{subsec:subsec1}






%--------------------------------------------------------------------------------------
% Insere a seção de cronograma
% Está comentada porque só é necessária no TCC I
%--------------------------------------------------------------------------------------

%\section{Cronograma} \label{sec:crono}

%A tabela \ref{tab:cronograma} mostra o cronograma de atividades a serem executadas para o TCC II, com base no calendário de 201X.Y da UNIVASF.

%\newpage
%\begin{table}[!thb]
%	%\huge
%    \centering
%    \caption{\label{tab:cronograma} Cronograma das atividades previstas para o TCC II}
%%    \begin{adjustbox}{max width=\textwidth}
%    \begin{tabular}{p{6.5cm}|c|c|c|c|c|c}
%    \toprule
%    \textbf{Atividade}                      & Nov & Dez & Jan & Fev & Mar & Abr \\ \hline
%    Implementar o banco de dados              & X    & X     &       &        &          &          \\ \hline
%    Desenvolver a API HTTP RESTful                      &   X   & X     &       &        &          &          \\ \hline
%    Implementar o serviço de captura de dados        &      &      & X     &   X     &          &          \\ \hline
%    Desenvolver a aplicação \textit{Web/mobile} para exibição dos dados         &      &      & X     &   X     &     X     &          \\ \hline
 %   Teste do sistema            &      &       &       &        & X        &          %\\ \hline
 %   Escrita do TCC II                       &   X   & X     & X     & X      & X        & X        \\ \hline
%   Defesa do TCC II                        &      &       &       &        &          & X       \\
%    \bottomrule
 %   \end{tabular}
 %   \end{adjustbox}
%    \legend{\textbf{Fonte:} O autor.}
%\end{table}


		\chapter{Resultados} \label{ch:RD}

\begin{comment}
\section{Seção de exemplo 1 - Códigos} \label{sec:resex1}

\subsection{Subseção de exemplo 1 - Inserindo trechos de códigos}
 
O nosso querido Leonardo Cavalcante providenciou um comando que deixa nossos trechos de códigos bonitinhos e gera um elemento pré-textual de Lista de Códigos. 

Os códigos são adicionados através do comando seguinte:

\textbackslash sourcecode\{ Descrição \}\{Label\}\{Linguagem\}\{Arquivo com extensão\}

Um exemplo pode ser visto no código \ref{cmd:cron} abaixo.

\sourcecode{Configuração do intervalo de execução no Script Agendador}{cron}{javascript}{cron.js}


\section{Seção de exemplo 2 - Listas} \label{sec:resex2}

\subsection{Subseção de exemplo 2 - Lista de itens} 

Existem alguns tipos de listas no Latex, iremos exemplificar a lista sem numeração (seção \ref{subsubsec:itemize}), a lista enumerada (seção \ref{subsubsec:enumerate}) e a lista mista (seção \ref{subsubsec:mista}). As listas podem ser encadeadas de diversas maneiras,
de acordo com a necessidade do autor.

\subsubsection{Subsubseção de exemplo 1 - Lista sem numeração} \label{subsubsec:itemize}

Este é um exemplo de lista sem numeração.

\begin{itemize}
	\item \textbf{Cadastrar usuário}

		\begin{itemize}
    		\item Atores
		    	\begin{itemize}
    		    	\item Usuário
		    	\end{itemize}

	    	\item Fluxo de eventos primário
			    \begin{itemize}
	    		    \item o usuário deve se cadastrar informando seu nome, \textit{e-mail} e senha;
		        	\item a API armazena os dados do usuário;
		    	    \item o usuário é liberado para realizar o \textit{login}.
			    \end{itemize}

    		\item Fluxo alternativo
			    \begin{itemize}
		    	   \item o usuário desiste de se cadastrar e cancela o caso de uso clicando no botão voltar.
	    		\end{itemize}

		\end{itemize}
	
\end{itemize}

\subsubsection{Subsubseção de exemplo 2 - Lista enumerada} \label{subsubsec:enumerate}

Este é um exemplo de lista enumerada.

\begin{enumerate}
	\item O Usuário deseja ver o histórico das variáveis climáticas, então através da interface de usuário escolhe o período ao qual o histórico se refere;
	\item A aplicação solicita à API através de uma requisição HTTP contendo o momento de início e o momento do fim do período em seus parâmetros;     			\item A API recebe a solicitação e se comunica com a base de dados, então requere as informações quem possuem a data de leitura no intervalo escolhido;
	\item A base de dados retorna os dados em formato Json para a API;
	\item A API responde à requisição retornando os dados, também em formato Json, para a aplicação cliente;
	\item A aplicação cliente renderiza os gráficos utilizando o conjunto de dados obtidos.
\end{enumerate}

\subsubsection{Subsubseção de exemplo 3 - Lista mista} \label{subsubsec:mista}

Este é um exemplo de lista mista.

\begin{itemize}
	\item \textbf{Cadastrar usuário}

		\begin{itemize}
    		\item Atores
		    	\begin{itemize}
    		    	\item Usuário
		    	\end{itemize}

	    	\item Fluxo de eventos primário
			    \begin{enumerate}
	    		    \item o usuário deve se cadastrar informando seu nome, \textit{e-mail} e senha;
		        	\item a API armazena os dados do usuário;
		    	    \item o usuário é liberado para realizar o \textit{login}.
			    \end{enumerate}

    		\item Fluxo alternativo
			    \begin{itemize}
		    	   \item o usuário desiste de se cadastrar e cancela o caso de uso clicando no botão voltar.
	    		\end{itemize}

		\end{itemize}

	\item \textbf{Visualizar dados atuais}

		\begin{itemize}
		    \item Atores
	    		\begin{itemize}
		    	    \item Usuário
			    \end{itemize}
    
	    	\item Pré-condições
			    \begin{itemize}
		     	   \item o usuário deve estar autenticado
			    \end{itemize}

	    	\item Fluxo de eventos primário
			    \begin{enumerate}
		    	    \item o usuário deve efetuar o \textit{login} informando o \textit{e-mail} e a senha;
	    		    \item caso o usuário não seja autenticado, o sistema informa a respeito de credenciais inválidas e encerra o caso de uso;
		    	    \item a API autentica o usuário;
    			    \item o usuário é liberado para visualizar os dados atuais dos sensores da estação;
		        	\item após a visualização o usuário pode finalizar o caso de uso ou efetuar uma nova consulta se desejar.
			    \end{enumerate}

    		\item Fluxo alternativo
			    \begin{itemize}
    			   \item o usuário desiste de visualizar os dados atuais e cancela o caso de uso clicando no botão voltar.
			    \end{itemize}

		\end{itemize}

	\item \textbf{Visualizar histórico}

		\begin{itemize}
		    \item Atores
	    		\begin{itemize}
		    	    \item Usuário
	    		\end{itemize}

	    	\item Pré-condições
    			\begin{itemize}
			        \item o usuário deve estar autenticado
			    \end{itemize}

		    \item Fluxo de eventos primário
			    \begin{enumerate}
			        \item o usuário deve efetuar o \textit{login} informando o \textit{e-mail} e a senha;
			        \item caso o usuário não seja autenticado, o sistema informa a respeito de credenciais inválidas e encerra o caso de uso;
			        \item a API autentica o usuário;
			        \item o usuário é liberado para escolher qual período cujo histórico será exibido;
			        \item o usuário seleciona as variáveis a serem exibidas no gráficos de linhas;
			        \item após a visualização do histórico o usuário pode finalizar o caso de uso se desejar.
			    \end{enumerate}

		    \item Fluxo alternativo
			    \begin{enumerate}
			        \item após a escolha do período de exibição do histórico o usuário pode voltar para a tela anterior e escolher um novo período;
			        \item o histórico é exibido para o usuário;
			        \item após a visualização do histórico o usuário pode finalizar o caso de uso ou efetuar uma nova consulta se desejar.
			    \end{enumerate}

		    \item Fluxo alternativo
			    \begin{enumerate}
			        \item o usuário desiste de visualizar o histórico e cancela o caso de uso clicando no botão voltar.
			    \end{enumerate}
		\end{itemize}
\end{itemize}
\end{comment}



		%--------------------------------------------------------------------------------------
% Este arquivo contém a sua conclusão
%--------------------------------------------------------------------------------------
\chapter{Considerações Finais e Trabalhos Futuros}

%\lipsum[1-2]
 
\section{Trabalhos futuros}

%\lipsum[55]

	\postextual
		\bibliography{tex/references}
		\begin{anexosenv}




































\begin{comment}
\chapter{Comandos seriais da estação meteorológica \textit{Vantage Vue™}} \label{anex:anexo1}

\begin{center}
\scalefont{0.85}
\begin{longtable}{ll}
\caption{Comandos seriais suportados pela estação meteorológica \textit{Vantage Vue™}}\\
\hline
\multicolumn{1}{c}{\textbf{Instrução}} & \multicolumn{1}{c}{\textbf{Descrição}} \\ \hline
\endfirsthead

\multicolumn{2}{c}%
{{\bfseries \tablename\ \thetable{} -- Continuação da página anterior}} \\

\hline
\multicolumn{1}{c}{\textbf{Instrução}} & \multicolumn{1}{c}{\textbf{Descrição}} \\ \hline
\endhead

\multicolumn{2}{r}{{Continua na próxima página}} \\
\endfoot

\endlastfoot

\multicolumn{2}{c}{\cellcolor{gray!25}\textbf{Comandos de teste}}                                                   		 \\ \hline
\textbf{TESTE}                            & Envia a \textit{string} "TEST\textbackslash n" de volta  \\ \hline
\textbf{WRD}                        & Responde com o tipo de estação meteorológica \\ \hline
\textbf{RXCHECK}                        & Responde com o diagnóstico do Console \\ \hline
\textbf{RXTEST}                       & Muda a tela do console de \textit{"Receiving from"} para tela de dados atuais                                                        \\ \hline
\textbf{VER}                           & Responde com a data do \textit{firmware}                                                             \\ \hline
\textbf{RECEIVERS}                    & Responde com a lista das estações que o console "enxerga" \\ \hline
\textbf{NVER}                       & Responde com a versão do \textit{firmware}                                                             \\ \hline
\multicolumn{2}{c}{\cellcolor{gray!25}\textbf{Comandos de dados atuais}}                                             \\ \hline
\textbf{LOOP}                     & Responde com a quantidade de pacotes especificada a cada 2s        \\ \hline
\textbf{LPS}                & Responde a cada 2s com a quantidade de pacotes diferentes especificada          \\ \hline
\textbf{HILOWS}                & Responde com todo os dados de \textit{high/low}                 \\ \hline
\textbf{PUTRAIN}                      & Seta a quantidade anual de precipitação \\ \hline
\textbf{PUTET}                 & Seta a quantidade anual de evapotranspiração        \\ \hline
\multicolumn{2}{c}{\cellcolor{gray!25}\textbf{Comandos de \textit{download}}}                                     		 \\ \hline
\textbf{DMP}                 & Faz o \textit{download} de todo o arquivo de memória \\ \hline
\textbf{DMAFT}                   & Faz o \textit{download} de todo o arquivo de memória após a data especificada \\ \hline
\multicolumn{2}{c}{\cellcolor{gray!25}\textbf{Comandos da EEPROM}}                                     		 \\ \hline
\textbf{GETEE}                 & Lê toda a memória EEPROM \\ \hline
\textbf{EEWR}                   & Escreve um \textit{byte} de dados à partir do endereço especificado                                   \\ \hline
\textbf{EERD}                   & Lê a quantidade de dados especificada iniciando no endereço especificado                                   \\ \hline
\textbf{EEBWR}                   & Escreve os dados na EEPROM                                    \\ \hline
\textbf{EEBRD}                   & Lê os dados da EEPROM \\ \hline
\multicolumn{2}{c}{\cellcolor{gray!25}\textbf{Comandos de calibração}}                                     		 \\ \hline
\textbf{CALED}                 & Envia os dados da temperatura e umidade corrente para atribuir à calibração \\ \hline
\textbf{CALFIX}                   & Atualiza o \textit{display} quando os números de calibração mudam\\ \hline
\textbf{BAR}                   & Seta os valores da elevação e o \textit{offset} do barômetro quando a localização é alterada                                   \\ \hline
\textbf{BARDATA}                   & Mostra os valores atuais da calibração do barômetro                                   \\ \hline \\
\multicolumn{2}{c}{\cellcolor{gray!25}\textbf{Comandos de limpeza}}                                     		 \\ \hline
\textbf{CLRLOG}                 & Limpa todo o arquivo de dados                                                       \\ \hline
\textbf{CLRALM}                   & Limpa todos os limiares dos alarmes                                   \\ \hline
\textbf{CLRCAL}                   & Limpa todos os \textit{offsets} da calibração da temperatura e da umidade \\ \hline
\textbf{CLRGRA}                   & Limpa o gráfico do console \\ \hline
\textbf{CLRVAR}                   & Limpa o valor da precipitação ou da evapotranspiração \\ \hline
\textbf{CLRHIGHS}                   & Limpa todos os valores de pico diários, mensais ou anuais                                   \\ \hline
\textbf{CLRLOWS}                   & Limpa todos os valores de mínimos diários, mensais ou anuais \\ \hline
\textbf{CLRBITS}                   & Limpa os \textit{bits} de alarme ativos                                  \\ \hline
\textbf{CLRDATA}                   & Limpa todos os dados atuais                                   \\ \hline
\multicolumn{2}{c}{\cellcolor{gray!25}\textbf{Comandos de configuração}}                                     		 \\ \hline
\textbf{BAUD}                 & Atribui o valor do \textit{baudrate} do console                                                       \\ \hline
\textbf{SETTIME}                   & Define a data e a hora do console                                   \\ \hline
\textbf{GAIN}                   & Define o ganho do receptor de rádio                                   \\ \hline
\textbf{GETTIME}                   & Retorna a hora e a data atual do console                                   \\ \hline
\textbf{SETPER}                   & Define o intervalo de arquivamento                                   \\ \hline
\textbf{STOP}                   & Desabilita a criação dos registros                                   \\ \hline
\textbf{START}                   & Habilita a criação dos arquivos \\ \hline
\textbf{NEWSETUP}                   & Reinicia o console após alguma configuração nova                                  \\ \hline
\textbf{LAMPS}                   & Liga ou desliga as lâmpadas do console \\ \hline

%\label{tab:6}
\end{longtable}
\fonte{\citeonline{VSPDOC} (Traduzido).}
\end{center}


\end{anexosenv}



\end{comment}



\end{document}
